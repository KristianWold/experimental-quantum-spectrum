%%%%%%%%%%%%%%%%%%%%%%%%%%%%%%%%%%%%%%%%%%%%%%%%%%%%%%%%%%%%%%%%%%%%%%
%%%    Sample abstract for CDNBQS 2023                                    
%%%                                                                      
%%%    Please, edit this file as required.                             
%%%                                                                    
%%%    Submit your abstract via e-mail to cdnbqs@uevora.pt      
%%%    with the subject line,                                          
%%%    "CDNBQS 2023 Abstract Submission yourlastname".                  
%%%    Confirmation of your abstract submission                        
%%%    will be sent to you by e-mail.                                  
%%%                                                                    
%%%    If you would like to compile the latex source, add              
%%%         \documentclass[a5paper,aps,11pt]{revtex4}                 
%%%         \begin{document} 
%%%    or      
%%%         \documentclass[a5paper,aps,11pt]{revtex4-1}                 
%%%         \begin{document}                                         
%%%    at the beginning of the file and                            
%%%         \maketitle                                               
%%%         \end{document}                                           
%%%    at the end.    
%%%    But remove these commands before submitting the abstract.                                                 
%%%    The figures section should be commented before compiling.      
%%%%%%%%%%%%%%%%%%%%%%%%%%%%%%%%%%%%%%%%%%%%%%%%%%%%%%%%%%%%%%%%%%%%%%
%%%%%%%%%%%%%%%%%%%%%%%%%%%%%%%%%%%%%%%%%%%%%%%%%%%%%%%%%%%%%%%%%%%%%%
%%%    TITLE:                               
%%%%%%%%%%%%%%%%%%%%%%%%%%%%%%%%%%%%%%%%%%%%%%%%%%%%%%%%%%%%%%%%%%%%%%

\title{Universal spectral properties of noisy intermediate scale quantum circuits}

%%%%%%%%%%%%%%%%%%%%%%%%%%%%%%%%%%%%%%%%%%%%%%%%%%%%%%%%%%%%%%%%%%%%%%
%%%    AUTHOR(S) AND AFFILIATION(S):                               
%%%                                                                
%%%    For more authors with a different affiliation               
%%%    add more \index{}, \author{} and \affiliation{} lines.      
%%%    Affiliation lines must end with \\, except for the last line  
%%%                                                                
%%%%%%%%%%%%%%%%%%%%%%%%%%%%%%%%%%%%%%%%%%%%%%%%%%%%%%%%%%%%%%%%%%%%%%
\index{K. Wold}
\index{P. Ribeiro}
\index{S. Denisov}

\author{K. Wold$^{1,2}$, P. Ribeiro$^3$, S. Denisov$^1$}
\affiliation{1~Department of Computer Science, OsloMet – Oslo Metropolitan University, N-0130 Oslo, Norway \\
             2~NordSTAR – Nordic Center for Sustainable and Trustworthy AI Research, N-0166 Oslo, Norway \\
             3~CeFEMA, Instituto Superior Técnico, Universidade de Lisboa, 1049-001 Lisboa, Portugal}

%%%%%%%%%%%%%%%%%%%%%%%%%%%%%%%%%%%%%%%%%%%%%%%%%%%%%%%%%%%%%%%%%%%%%%
%%%    ABSTRACT:                            
%%%    End with two "\newline" commands.    
%%%%%%%%%%%%%%%%%%%%%%%%%%%%%%%%%%%%%%%%%%%%%%%%%%%%%%%%%%%%%%%%%%%%%%
\begin{abstract}
Present-day noisy intermediate-scale quantum (NISQ) computing platforms allow for implementing unitary circuits only on very limited size- and time-scales. While universal properties of random
unitary circuits have been analyzed in detail, the question what happens to these unversilaties in the
NISQ realm remains open. To answer the question, we implement different variational circuits on
the IBM Quantum platform and model these implementations as quantum channels. To find parameters of the channels, we perform series of tomography-like experiments and process the results with
a machine learning algorithm. We demonstrate that the spectra of the recovered channels exhibit
universal properties typical to a recently introduced ensemble of quantum maps, which allows for a
complete analytical evaluation. Our results establish novel connections between NISQ computing,
machine learning, and random matrix theory, and highlight the present-day quantum computer
prototypes as flexible experimental platforms to explore phenomena of Dissipative Quantum Chaos.
\newline
\newline
%%%%%%%%%%%%%%%%%%%%%%%%%%%%%%%%%%%%%%%%%%%%%%%%%%%%%%%%%%%%%%%%%%%%%%
%%%    Include here a figure if needed.     
%%%    Remove "%" to uncomment.              
%%%    Otherwise, delete.                   
%%%%%%%%%%%%%%%%%%%%%%%%%%%%%%%%%%%%%%%%%%%%%%%%%%%%%%%%%%%%%%%%%%%%%%
%   \begin{center}
%     \includegraphics[width=0.4\textwidth]{bcs.eps}
%   \end{center}
%%%%%%%%%%%%%%%%%%%%%%%%%%%%%%%%%%%%%%%%%%%%%%%%%%%%%%%%%%%%%%%%%%%%%%
%%%    Include here the caption of the figure.
%%%    End with two "\newline" commands.    
%%%    If not needed, delete.               
%%%%%%%%%%%%%%%%%%%%%%%%%%%%%%%%%%%%%%%%%%%%%%%%%%%%%%%%%%%%%%%%%%%%%%
%   \caption{Fig.~1:  Energy gap for single-particle-like excitations  versus temperature.}
%   \newline
%   \newline
%%%%%%%%%%%%%%%%%%%%%%%%%%%%%%%%%%%%%%%%%%%%%%%%%%%%%%%%%%%%%%%%%%%%%%
%%%    Include here the bibliography.       
%%%    Each line must end with \newline     
%%%    except for the last line.            
%%%%%%%%%%%%%%%%%%%%%%%%%%%%%%%%%%%%%%%%%%%%%%%%%%%%%%%%%%%%%%%%%%%%%%
%1.  H. Frohlich, Phys. Rev. {\bf  79}, 845 (1950). \newline
%2.  L. N. Cooper, Phys. Rev. {\bf 104}, 1189 (1956) .

\end{abstract}